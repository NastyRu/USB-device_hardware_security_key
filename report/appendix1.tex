\chapter{Листинги}

 \begin{lstlisting}[caption = Структура usb\_device, label =  lst:usb_device]
struct usb_device {
	int		devnum;
	char		devpath[16];
	u32		route;
	enum usb_device_state	state;
	enum usb_device_speed	speed;
	unsigned int		rx_lanes;
	unsigned int		tx_lanes;

	struct usb_tt	*tt;
	int		ttport;

	unsigned int toggle[2];

	struct usb_device *parent;
	struct usb_bus *bus;
	struct usb_host_endpoint ep0;

	struct device dev;

	struct usb_device_descriptor descriptor;
	struct usb_host_bos *bos;
	struct usb_host_config *config;

	struct usb_host_config *actconfig;
	struct usb_host_endpoint *ep_in[16];
	struct usb_host_endpoint *ep_out[16];

	char **rawdescriptors;

	unsigned short bus_mA;
	u8 portnum;
	u8 level;
	u8 devaddr;

	unsigned can_submit:1;
	unsigned persist_enabled:1;
	unsigned have_langid:1;
	unsigned authorized:1;
	unsigned authenticated:1;
	unsigned wusb:1;
	unsigned lpm_capable:1;
	unsigned usb2_hw_lpm_capable:1;
	unsigned usb2_hw_lpm_besl_capable:1;
	unsigned usb2_hw_lpm_enabled:1;
	unsigned usb2_hw_lpm_allowed:1;
	unsigned usb3_lpm_u1_enabled:1;
	unsigned usb3_lpm_u2_enabled:1;
	int string_langid;

	/* static strings from the device */
	char *product;
	char *manufacturer;
	char *serial;

	struct list_head filelist;

	int maxchild;

	u32 quirks;
	atomic_t urbnum;

	unsigned long active_duration;

#ifdef CONFIG_PM
	unsigned long connect_time;

	unsigned do_remote_wakeup:1;
	unsigned reset_resume:1;
	unsigned port_is_suspended:1;
#endif
	struct wusb_dev *wusb_dev;
	int slot_id;
	enum usb_device_removable removable;
	struct usb2_lpm_parameters l1_params;
	struct usb3_lpm_parameters u1_params;
	struct usb3_lpm_parameters u2_params;
	unsigned lpm_disable_count;

	u16 hub_delay;
	unsigned use_generic_driver:1;
};
 \end{lstlisting}
 
  \begin{lstlisting}[caption = Структура usb\_device\_id, label =  lst:usb_device_id]
 struct usb_device_id {
	/* which fields to match against? */
	__u16		match_flags;

	/* Used for product specific matches; range is inclusive */
	__u16		idVendor;
	__u16		idProduct;
	__u16		bcdDevice_lo;
	__u16		bcdDevice_hi;

	/* Used for device class matches */
	__u8		bDeviceClass;
	__u8		bDeviceSubClass;
	__u8		bDeviceProtocol;

	/* Used for interface class matches */
	__u8		bInterfaceClass;
	__u8		bInterfaceSubClass;
	__u8		bInterfaceProtocol;

	/* Used for vendor-specific interface matches */
	__u8		bInterfaceNumber;

	/* not matched against */
	kernel_ulong_t	driver_info
		__attribute__((aligned(sizeof(kernel_ulong_t))));
};
 \end{lstlisting}

  \begin{lstlisting}[caption = Добавление usb устройства, label =  lst:add_usb]
static void add_our_usb_device(struct usb_device *dev)
{
    our_usb_device_t* new_usb_device = (our_usb_device_t *)kmalloc(sizeof(our_usb_device_t), GFP_KERNEL);
    struct usb_device_id new_id = { USB_DEVICE(dev->descriptor.idVendor, dev->descriptor.idProduct) };
    new_usb_device->dev_id = new_id;
    list_add_tail(&new_usb_device->list_node, &connected_devices);
}
 \end{lstlisting}
 
  \begin{lstlisting}[caption = Удаление usb устройства, label =  lst:del_usb]
static void delete_our_usb_device(struct usb_device *dev)
{
    our_usb_device_t *device, *temp;
    list_for_each_entry_safe(device, temp, &connected_devices, list_node) 
    {
        if (device_match_device_id(dev, &device->dev_id))
        {
            list_del(&device->list_node);
            kfree(device);
        }
    }
}
 \end{lstlisting}
 
  \begin{lstlisting}[caption = Makefile, label =  lst:makefile]
ifneq ($(KERNELRELEASE),)
	obj-m := md.o
else
	CURRENT = $(shell uname -r)
	KDIR = /lib/modules/$(CURRENT)/build
	PWD = $(shell pwd)

default:
	$(MAKE) -C $(KDIR) M=$(PWD) modules

clean:
	rm -rf .tmp_versions
	rm *.ko
	rm *.o
	rm *.mod.c
	rm *.symvers
	rm *.order

endif
 \end{lstlisting}
 
   \begin{lstlisting}[caption = Конфигурационный файл USB устройств, label =  lst:config_usb]
 struct known_usb_device {
    struct usb_device_id dev_id;
    char *name;
};

// List of all USB devices you know
static const struct known_usb_device known_devices[] = {
    { .dev_id = { USB_DEVICE(0x058f, 0x6387) }, .name = "SAG" },
};
  \end{lstlisting}

   \begin{lstlisting}[caption = Конфигурационный файл секретных файлов и приложений, label =  lst:config_file]  
  static char *secret_apps[] = {
    "/home/parallels/Desktop/Operating_systems_coursework/file.txt",
    "/home/parallels/Desktop/Operating_systems_coursework/xor",
    "/usr/bin/firefox",
	NULL,
};
    \end{lstlisting}
 
   \begin{lstlisting}[caption = Загрузка и удаление модуля ядра, label =  lst:module]
static int __init my_module_init(void)
{
    usb_register_notify(&usb_notify);
    call_encryption();
    printk(KERN_INFO "USB MODULE: loaded.\n");
    return 0;
}

static void __exit my_module_exit(void)
{
    usb_unregister_notify(&usb_notify);    
    printk(KERN_INFO "USB MODULE: unloaded.\n");
}

module_init(my_module_init);
module_exit(my_module_exit);
 \end{lstlisting}
 
 
    \begin{lstlisting}[caption = Функция-обработчик, label =  lst:notify]
// If usb device inserted.
static void usb_dev_insert(struct usb_device *dev)
{   
    add_our_usb_device(dev);
    char *name = knowing_device();
    
    if (name)
    {
        if (state_encrypt)
            call_decryption(name);
        state_encrypt = false;
        printk(KERN_INFO "USB MODULE: New device we can encrypt.\n");
    }
    else
   {
        if (!state_encrypt)
            call_encryption();
        state_encrypt = true;
        printk(KERN_INFO "USB MODULE: New device, we can't encrypt.\n");
    }
}

// If usb device removed.
static void usb_dev_remove(struct usb_device *dev)
{
    delete_our_usb_device(dev);
    char *name = knowing_device();

    if (name)
    {
        if (state_encrypt)
            call_decryption(name);
        state_encrypt = false; 
        printk(KERN_INFO "USB MODULE: Delete device, we can encrypt.\n");
    }
    else
    {
        if (!state_encrypt)
            call_encryption();
        state_encrypt = true;
        printk(KERN_INFO "USB MODULE: Delete device, we can't encrypt.\n");
    }
}

// New notify.
static int notify(struct notifier_block *self, unsigned long action, void *dev)
{
    // Events, which our notifier react.
    switch (action) 
    {
        case USB_DEVICE_ADD:
            usb_dev_insert(dev);
	        break;
        case USB_DEVICE_REMOVE:
            usb_dev_remove(dev);
	        break;
        default:
	        break;
    }
    return 0;
}
 \end{lstlisting}
 
\begin{lstlisting}[caption = Функции для проверки разрешенных устройств, label =  lst:check]
// Match device id with device id.
static bool device_id_match_device_id(struct usb_device_id *new_dev_id, const struct usb_device_id *dev_id)
{
    // Check idVendor and idProduct, which are used.
    if (dev_id->idVendor != new_dev_id->idVendor)
        return false;
    if (dev_id->idProduct != new_dev_id->idProduct)
        return false;
    return true;
}

// Check our list of devices, if we know device.
static char *usb_device_id_is_known(struct usb_device_id *dev)
{
    unsigned long known_devices_len = sizeof(known_devices) / sizeof(known_devices[0]);
    int i = 0;
    for (i = 0; i < known_devices_len; i++)
    {
        if (device_id_match_device_id(dev, &known_devices[i].dev_id))
        {
            int size = sizeof(known_devices[i].name);
            char *name = (char *)kmalloc(size + 1, GFP_KERNEL);
            int j = 0;
            for (j = 0; j < size; j++)
                name[j] = known_devices[i].name[j];
            name[size + 1] = '\0';

            return name;
        }
    }
    return NULL;
}

static char *knowing_device(void)
{
    our_usb_device_t *temp;
    int count = 0;
    char *name;

    list_for_each_entry(temp, &connected_devices, list_node) {
        name = usb_device_id_is_known(&temp->dev_id);
        if (!name)
            return NULL;
        count++;
    }
    if (0 == count)
        return NULL;
    return name;
}
 \end{lstlisting}
 
 
    \begin{lstlisting}[caption = Считывание пароля из файла USB устройства, label =  lst:password]
static char *read_file(char *filename)
{
    struct kstat *stat;
    struct file *fp;
    mm_segment_t fs;
    loff_t pos = 0;
    char *buf;
    int size;

    fp = filp_open(filename, O_RDWR, 0644);
    if (IS_ERR(fp))
    {
        return NULL;
    }

    fs = get_fs();
    set_fs(KERNEL_DS);
    
    stat = (struct kstat *)kmalloc(sizeof(struct kstat), GFP_KERNEL);
    if (!stat)
    {
        return NULL;
    }

    vfs_stat(filename, stat);
    size = stat->size;

    buf = kmalloc(size, GFP_KERNEL);
    if (!buf) 
    {
        kfree(stat);
        return NULL;
    }

    kernel_read(fp, buf, size, &pos);

    filp_close(fp, NULL);
    set_fs(fs);
    kfree(stat);
    buf[size]='\0';
    return buf;
}
 \end{lstlisting}
 
 \begin{lstlisting}[caption = Функции вызывающие исполняемый файл пользовательского пространства, label =  lst:user_call]
 static int call_decryption(char *name_device) {
	printk(KERN_INFO "USB MODULE: Call_decrypt\n");
    
    char path[80];
    strcpy(path, USB_FOLDER);
    strcat(path, name_device);
    strcat(path, "/");
    strcat(path, PASSWORD_FILE);
    char *data = read_file(path);
	
    char *argv[] = {
     "/home/parallels/Desktop/Operating_systems_coursework/crypto",
    data,
    NULL };

    static char *envp[] = {
    "HOME=/",
    "TERM=linux",
    "PATH=/sbin:/bin:/usr/sbin:/usr/bin", 
     NULL };

    if (call_usermodehelper(argv[0], argv, envp, UMH_WAIT_PROC) < 0) 
    {
        return -1;
    }

    return 0;
}

static int call_encryption(void) {
    printk(KERN_INFO "USB MODULE: Call_encrypt\n");
    char *argv[] = {
     "/home/parallels/Desktop/Operating_systems_coursework/crypto",
    NULL };

    static char *envp[] = {
    "HOME=/",
    "TERM=linux",
    "PATH=/sbin:/bin:/usr/sbin:/usr/bin", 
    NULL };

    if (call_usermodehelper(argv[0], argv, envp, UMH_WAIT_PROC) < 0) 
    {
        return -1;
    }

    return 0;
}
  \end{lstlisting}
%%% Local Variables: 
%%% mode: latex
%%% TeX-master: "rpz"
%%% End: 
